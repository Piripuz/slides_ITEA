\documentclass[hyperref={pdfpagelabels=false}]{beamer}
\usepackage{helvet}
\usepackage[english]{babel}
\usepackage{pgfplots}
\usepackage{pgf}
\pgfplotsset{compat=newest}
\usepackage{booktabs}
\usepackage[T1]{fontenc}
\usepackage[utf8]{inputenc}
\usepackage{lipsum}
\usepackage{tcolorbox}
\usepackage{xcolor}
\usepackage{listings}
\usepackage[nopatch=footnote]{microtype}
\usepackage{float}
\usepackage{siunitx}
\usepackage{multicol}
\usepackage{hyperref}
\usepackage{dsfont}
\usepackage{caption}
\usepackage{subcaption}
\usepackage[backend=biber,style=authoryear-comp,sorting=nyt]{biblatex} % Package for bibliography (citing)
\bibliography{bibliography.bib}
\graphicspath{{./img}}

\input{template/settings}

\newcommand{\setcolor}[1]{\def\chosencolor{#1}}
\newcommand{\setdepartment}[1]{\def\department{#1}}

\setbeamerfont{caption}{size=\tiny}

% \setbeamersize{text margin left=22mm}

\usetikzlibrary{shapes, shapes.callouts, positioning, arrows.meta, calc}
\usepackage{eurosym}
\usepackage{animate}

%%% Local Variables:
%%% mode: LaTeX
%%% TeX-master: "main"
%%% End:

% \includeonlyframes{current}
\title{Reverse ADL-Vickrey}
\subtitle{Inferring Scheduling Preferences based on Congestion and Choice Data}
\author{Pietro Giardina}

\setdepartment{DTU Compute}
\setcolor{dtured}



\begin{document}
\inserttitlepage


\section{Introduction} % Vickrey
\begin{frame}
  \tableofcontents[currentsection]
\end{frame}

\begin{frame}
  \frametitle{Is Congestion a Problem?}
  According to a recent study\footcite{kim2022congestion},
  \begin{itemize}
  \item An \textbf{average} delay of 4-5 minutes \textbf{per commuter} is caused by congestion
  \item This damage is worth, in California, \textbf{1.67\%} of the state GDP.
  \end{itemize}
\end{frame}

\begin{frame}
  \frametitle{How is congestion created?}
  \begin{figure}
    \centering
    \includegraphics[width=.7\textwidth]{travel_time}
    \caption{Travel Time, on California State Route 237, the 16th of February 2018.
    Data from CalTrans PeMS.}
    \label{fig:tt}
  \end{figure}
  \begin{itemize}
  \item All commuters want to travel at the same time, with the same mode
  \item The infrastructure saturates beyond capacity
  \end{itemize}
\end{frame}

\begin{frame}
  \frametitle{How can congestion be tackled?}
  On demand side, Travel Demand Management:
  \begin{itemize}
  \item Influencing mode choice
  \item Influencing route/destination choice
  \item Incentivize spreading of departure times
  \end{itemize}

  How can we have an impact on when people decide to leave?
\end{frame}

\begin{frame}
  \frametitle{When do People Travel?}
  According to the bottleneck model\footcite{f32d6720-dd02-34b7-a4ba-c4c21193efe7, d0907f84-e14a-3d98-ad20-759f41491d6e},
  minimising three main discomforts:
  \begin{itemize}
  \item Travel time
  \item Arriving too early
  \item Arriving too late
  \end{itemize}

  The cost is assumed to grow linearly on these three components
\end{frame}

\begin{frame}
  \frametitle{The \(\alpha\)-\(\beta\)-\(\gamma\) Model}
  Each commuter minimizes a cost function (from \cite{d0907f84-e14a-3d98-ad20-759f41491d6e})
  \begin{equation*}
    C(t_d) = \alpha(\text{travel time}) + \beta (\text{time early}) + \gamma (\text{time late})
  \end{equation*}
  \begin{figure}
    \centering
    \includegraphics[width=.7\textwidth]{cost_simple}
  \end{figure}
\end{frame}

\begin{frame}
  \frametitle{Calibrating the Parameters}
  For many applications, it is important to find values for \(\alpha\), \(\beta\) and \(\gamma\) (\textit{scheduling preferences}) and \(t^*\).
  \begin{itemize}
  \item Historically, estimation was done with surveys
  \item \textcite{54d203ee-4bf8-3234-9286-56e4c8b7f5bd} estimates the following:
    \begin{equation*}
      \frac{\beta}{\alpha} = 0.61\qquad \frac{\gamma}{\alpha} = 2.38
    \end{equation*}
  \item \textcite{https://doi.org/10.1111/iere.12692} presents a brief review of the results
  \end{itemize}
\end{frame}

\begin{frame}
  \frametitle{Weaknesses of Surveys}
  Surveys are extremely costly, and can suffer from many sources of error
  \begin{itemize}
  \item Answers may be biased, due to reasons beyond the control of the scientist
  \item Rounding may limit the precision
  \end{itemize}

  \begin{tikzpicture}[overlay, remember picture]
    \node[
    ellipse callout,
    callout relative pointer={(-1,-1)},
    fill=dtured!70,
    text width=5.5cm,
    align=center,
    above right,
    xshift=2cm,
    yshift=2cm
    ] (Q) at (current page.south west) {How much should I pay you for arriving 5 minutes earlier to work?};
    \node[
    ellipse callout,
    callout relative pointer={(.5,-.5)},
    fill=dtured!70,
    % text width=3cm,
    align=center,
    below right=.5cm of Q,
    xshift=1cm,
    % yshift=2cm
    ] (A) {\(x\) \euro};
  \end{tikzpicture}
\end{frame}
\begin{frame}
  \frametitle{Revealed Preferences data}
  Data about Revealed Preferences may offer some advantages
  \begin{itemize}
  \item A priori, more precise
  \item Can be systematically harvested
  \item More difficult to elaborate
  \end{itemize}
  (image about RP)
\end{frame}

\begin{frame}
  \frametitle{Research Question}
  Can we calibrate the parameters \(\alpha\), \(\beta\) and \(\gamma\) by using only Revealed Preferences?
\end{frame}

\section{Methodology}
\begin{frame}
  \tableofcontents[currentsection]
\end{frame}

\begin{frame}
  \frametitle{The Idea}
  From data that are objectively measurable,
  we estimate the values for scheduling preferences:
  
  \begin{figure}
    \centering
    \begin{tikzpicture}
      \node[
      align=center,
      ellipse,
      fill=dtured!30,
      font=\small
      ] (dep) {Individual\\ Departure Times};
      \node[
      align=center,
      ellipse,
      fill=dtured!30,
      font=\small,
      below = .7 cm of dep
      ] (tt) {Travel time,\\for each Time Point};
      \node[right = 2cm of {$(dep) !.5! (tt)$}] (mid) {};
      \draw[
      -{Triangle[width=32pt,length=23pt]},
      line width=20pt,
      dtured
      ] (mid) -- ++(2.0cm,0);
      \node[
      right=2.4 cm of mid,
      align=center,
      ellipse,
      fill=dtured!30,
      font=\small,
      ] {Original\\Parameters \\\(\alpha, \beta, \gamma, t^*\)};
    \end{tikzpicture}
  \end{figure}
\end{frame}

\begin{frame}
  \frametitle{Working Principle}
  The development of the model is done in three main steps:
  \begin{enumerate}
  \item The optimal departure time is theoretically studied
  \item A function, expressing the likelihood of a data point, is found
  \item Maximising the likelihood, the parameters are estimated
  \end{enumerate}
\end{frame}

\begin{frame}
  \frametitle{Evaluation of the Model}
  Progressively more complex evaluations are possible:
  \begin{enumerate}
  \item Evaluation on a Synthetic Dataset
  \item Evaluation of data from a Traffic Simulator
  \item Evaluation on Real Data
  \end{enumerate}
\end{frame}

\section{Model Developement}

\begin{frame}
  \tableofcontents[currentsection]
\end{frame}

\begin{frame}
  \frametitle{The Cost Function}
  \begin{figure}
    \centering
    % \animategraphics[palindrome, autoplay,controls={play, stop},width=.7\textwidth]{20}{animation_no_shades/frame_}{0}{79}
    \includegraphics[width=.7\textwidth]{animation_no_shades/frame_79}
  \end{figure}
  Three different types of minima:
  \begin{itemize}
  \item \textcolor{green}{Early minima}
  \item \textcolor{blue}{On-time minima}
  \item \textcolor{red}{Late minima}
  \end{itemize}
\end{frame}

\begin{frame}
  \frametitle{Characterization of minima}
  Location of minima depends on tangents to the travel time function, given \(\alpha, \beta\) and \(\gamma\):
  \begin{figure}
    \centering
    % \animategraphics[palindrome, controls={play, stop},width=.8\textwidth]{20}{animation_shades/frame_}{0}{79}
    \includegraphics[width=.8\textwidth]{animation_shades/frame_79}
  \end{figure}
\end{frame}

\begin{frame}
  \frametitle{Computation of Likelihood}
  \begin{itemize}
  \item Given the characterization, an expression for the likelihood can be found:
  \scalebox{0.62}{\parbox{.5\linewidth}{
        \begin{align*}
          f_{T_a}(t_a) =
          & \color<2->{blue} f_{T^*}(t_a)\int_{\beta_0(t_a)}^{\beta_\text{max}}f_\beta(b)\, db\int_{\gamma_0(t_a)}^{\gamma_\text{max}}f_\gamma(g)\, dg \\
          & \color<2->{green}+ f_\beta(tt_a'(t_a))[tt_a''(t_a)]^+\int_1^\infty f_\gamma(x) \left(F_{T^*}(\min(b_e(tt_a'(t_a)), t_s(tt_a'(t_a), x))) - F_{T^*}(t_a)\right) dx\  \\
          & \color<2->{red}+ f_\gamma(tt_a'(t_a))[tt_a''(t_a)]^+  \int_0^1f_\beta(x) \left(F_{T^*}(t_a) - F_{T^*}(\max(g_i(-tt_a'(t_a)), t_s(x, -tt_a'(t_a)))) \right) dx\
        \end{align*}
      }}
  \item<2-> The expression for the likelihood is a sum of three different terms
    \begin{itemize}
    \item<2-> A term for \textcolor{blue}{on-time arrivals}
    \item<2-> A term for \textcolor{green}{early arrivals}
    \item<2-> A term for \textcolor{red}{late arrivals}
    \end{itemize}
  \end{itemize}
\end{frame}

\begin{frame}
  \frametitle{Maximising the Likelihood}
  By using the expression for the likelihood,
  scheduling delay parameters can be estimated:
  \begin{itemize}
  \item \(\beta, \gamma, t^*\) are assumed to be distributed
  \item The distributions are parametrized by a parameter \(\theta\)
  \item The likelihood yields a function \(\mathcal{L} : \theta \mapsto f_{T_a}(t_a | \theta)\)
  \item Maximising the likelihood, an estimate \(\hat{\theta}\) is found
  \end{itemize}
\end{frame}

\section{Evaluation of the Model}

\begin{frame}
  \tableofcontents[currentsection]
\end{frame}

\begin{frame}
  \frametitle{Evaluation on Synthetic Data}
  Evaluation is performed on a dataset built accordingly to the bottleneck model
  \begin{itemize}
  \item A value for the parameter \(\theta\) is chosen
  \item \(N\) samples are drawn from the distributions of \(\beta, \gamma, t^*\)
  \item For each of the \(N\) triple of samples, the cost function is minimized
  \item The model is run on the resulting dataset, yielding an estimate \(\hat{\theta}\) of the parameter \(\theta\)
  \end{itemize}
\end{frame}

\begin{frame}
  \frametitle{Choosing the Parameter $\theta$}
  Suppose the user characteristics distributed as
  \begin{equation*}
    \beta \sim \mathcal{N}(\mu_\beta, \sigma) \quad \gamma \sim \mathcal{N}(\mu_\gamma, \sigma) \quad t^* \sim \mathcal{N}(\mu_t, \sigma_t)
  \end{equation*}
  \(\theta \in \mathbb{R}^5\) can be, for instance, chosen as follows:
  \begin{equation*}
    \theta = \begin{pmatrix}
      \mu_\beta \\
      \mu_\gamma \\
      \mu_t \\
      \sigma \\
      \sigma_t
    \end{pmatrix}
    =
    \begin{pmatrix}
      0.5 \\
      1.4 \\
      9.5 \\
      0.3 \\
      1
    \end{pmatrix}
  \end{equation*}
\end{frame}

\begin{frame}
  \frametitle{Sampling Arrival Times}
  Arrival times are found by minimizing the cost.
  \begin{itemize}
  \item Independently minimizing the cost for \(N = 100,000\) samples, the following distribution is found
  \item<2-> The theoretical likelihood, for the original value of \(\theta\), closely resembles the empirical distribution
  \end{itemize}
  \centering
  \alt<1>{\includegraphics[width=.65\textwidth]{hist_no_ll}}{\includegraphics[width=.65\textwidth]{hist_ll}}
\end{frame}

\begin{frame}
  \frametitle{Plotting the Likelihood}
  By fixing some of the components of \(\theta\), the likelihood can be plotted when varying the rest of them.

  For \only<1>{\(\sigma = 0.3\)}\only<2>{\(\sigma = 0.03\)}\only<3>{\(\sigma = 1.3\)}

  \centering
  \only<1>{\includegraphics[width=.8\textwidth]{contour_beautiful}}
  \only<2>{\includegraphics[width=.8\textwidth]{contour_ugly}}
  \only<3>{\includegraphics[width=.8\textwidth]{contour_flat}}
\end{frame}

\begin{frame}
  \frametitle{Retrieving the parameter \(\theta\)}
  Running an optimizer on the likelihood,
  the parameter \(\theta \in \mathbb{R}^5\) can be retrieved.
  \begin{itemize}
  \item Sufficient precision is achieved, even with gradient-free optimization techniques
  \item Depending on the value of the parameter \(\sigma\),
    relative error can grow as high as \(15\%\).
  \item When the variance \(\sigma\) is low, a grid search can be necessary for choosing the initial condition.
  \end{itemize}
\end{frame}

\begin{frame}
  \frametitle{Evaluation on Simulated Data}
  A framework was developed for validation on data from the traffic simulator METROPOLIS2.
  \begin{itemize}
  \item The simulator cannot achieve convergence without employing a logit choice model
  \item Our model assumes deterministic users
  \item Convergence cannot be achieved on data from a traffic simulator
  \end{itemize}
\end{frame}

\begin{frame}
  \frametitle{Evaluation on Real Data}
  On Real data, the model shows deeper problems
  \begin{itemize}
  \item \textcite{https://doi.org/10.1111/iere.12692} reveals problem with low steepness of real data
  \item Our independent studies confirm the problem
  \item Non-equilibrium studies would infer extremely low values for \(\beta, \gamma\).
  \end{itemize}
\end{frame}

\begin{frame}
  \frametitle{Low Slopes in Real Data}
  According to \textcite{https://doi.org/10.1111/iere.12692},
  observed slopes are way lower than what would be expected from equilibrium solutions
  \begin{figure}
    \centering
    \includegraphics[width=.6\textwidth]{slopes_hall}
    \caption{Actual travel time, compared with theoretical equilibrium solution. Image taken from \textcite{https://doi.org/10.1111/iere.12692}}
  \end{figure}
  \begin{itemize}
  \item The same problem shows in both urban and non-urban settings
  \end{itemize}
\end{frame}

\section{Discussion}

\begin{frame}
  \tableofcontents[currentsection]
\end{frame}

\begin{frame}
  \frametitle{Extension to Real Data}
  The main purpose of the model is being deployed on real data
  \begin{itemize}
  \item Simulated data show problem with logit
  \item Real data show as well problems with slow steepnesses
  \item These problems can be tackled one by one, and be object of future research
  \end{itemize}
\end{frame}

\begin{frame}[allowframebreaks]
  \frametitle{References}
  \printbibliography
\end{frame}

\end{document}
%%% Local Variables:
%%% mode: LaTeX
%%% TeX-master: t
%%% End:
